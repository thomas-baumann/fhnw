\documentclass[10pt]{article}

%Math
\usepackage{amsmath}
\usepackage{amsfonts}
\usepackage{amssymb}
\usepackage{amsthm}
\usepackage{ulem}
%\usepackage{stmaryrd} %f\UTF{00FC}r Blitz!

% PageStyle
\usepackage[ngerman]{babel} % deutsche Silbentrennung
\usepackage[utf8]{inputenc}
\usepackage{fancyhdr, graphicx}
\usepackage[scaled=0.92]{helvet}
\usepackage{enumitem}
\usepackage{parskip}
\usepackage[a4paper,top=2cm]{geometry}
\setlength{\textwidth}{17cm}
\setlength{\oddsidemargin}{-0.5cm}


% Shortcommands
\newcommand{\Bold}[1]{\textbf{#1}} %Boldface
\newcommand{\Kursiv}[1]{\textit{#1}} %Italic
\newcommand{\T}[1]{\text{#1}} %Textmode
\newcommand{\Nicht}[1]{\T{\sout{$ #1 $}}} %Streicht Shit durch

%Arrows
\newcommand{\lra}{\leftrightarrow} 
\newcommand{\ra}{\rightarrow}
\newcommand{\la}{\leftarrow}
\newcommand{\lral}{\longleftrightarrow}
\newcommand{\ral}{\longrightarrow}
\newcommand{\lal}{\longleftarrow}
\newcommand{\Lra}{\Leftrightarrow}
\newcommand{\Ra}{\Rightarrow}
\newcommand{\La}{\Leftarrow}
\newcommand{\Lral}{\Longleftrightarrow}
\newcommand{\Ral}{\Longrightarrow}
\newcommand{\Lal}{\Longleftarrow}

% Code listenings
\usepackage{color}
\usepackage{xcolor}
\usepackage{listings}
\usepackage{caption}
\DeclareCaptionFont{white}{\color{white}}
\DeclareCaptionFormat{listing}{\colorbox{gray}{\parbox{\textwidth}{#1#2#3}}}
\captionsetup[lstlisting]{format=listing,labelfont=white,textfont=white}
\lstdefinestyle{JavaStyle}{
 language=Java,
 basicstyle=\footnotesize\ttfamily, % Standardschrift
 numbers=left,               % Ort der Zeilennummern
 numberstyle=\tiny,          % Stil der Zeilennummern
 stepnumber=1,              % Abstand zwischen den Zeilennummern
 numbersep=5pt,              % Abstand der Nummern zum Text
 tabsize=2,                  % Groesse von Tabs
 extendedchars=true,         %
 breaklines=true,            % Zeilen werden Umgebrochen
 frame=b,         
 %commentstyle=\itshape\color{LightLime}, Was isch das? O_o
 %keywordstyle=\bfseries\color{DarkPurple}, und das O_o
 basicstyle=\footnotesize\ttfamily,
 stringstyle=\color[RGB]{42,0,255}\ttfamily, % Farbe der String
 keywordstyle=\color[RGB]{127,0,85}\ttfamily, % Farbe der Keywords
 commentstyle=\color[RGB]{63,127,95}\ttfamily, % Farbe des Kommentars
 showspaces=false,           % Leerzeichen anzeigen ?
 showtabs=false,             % Tabs anzeigen ?
 xleftmargin=17pt,
 framexleftmargin=17pt,
 framexrightmargin=5pt,
 framexbottommargin=4pt,
 showstringspaces=false      % Leerzeichen in Strings anzeigen ?
}
\lstdefinelanguage{XMLStyle}{
  basicstyle=\footnotesize\ttfamily,
  morestring=[b]",
  morestring=[s]{>}{<},
  morecomment=[s]{<?}{?>},
  stringstyle=\color{black},
  identifierstyle=\color[rgb]{0.0,0.0,0.6},
  keywordstyle=\color[rgb]{0.0,0.6,0.6},
  commentstyle=\color[rgb]{0.4,0.4,0.4}\upshape,
  morekeywords={xmlns,version,type}, % list your attributes here
  showstringspaces=false
}

%Config
\renewcommand{\headrulewidth}{0pt}
\setlength{\headheight}{15.2pt}
\let\stdsection\section{}
\renewcommand{\section}{\newpage\stdsection}
\usepackage{dirtree}

%Metadata
\fancyfoot[C]{}
\title{
	\vspace{5cm}
	Verteilte Systeme FS 13\\
	%\vspace{1cm}
	Übung 2
}
\author{Thomas Baumann}
\date{18. März 2013}


% hier beginnt das Dokument
\begin{document}

% Titelbild
\maketitle
\thispagestyle{fancy}

\newpage

% Inhaltsverzeichnis
\pagenumbering{Roman}
\tableofcontents	  	


\newpage
\setcounter{page}{1}
\pagenumbering{arabic}

% Inhalt Start
\section{Beschreibung}
Für die Übung 2 habe ich das zur Verfügung gestellte Java GUI für den Clienten verwendet. 
Auf der Serverseite implementierte ich ein Servlet für einen Tomcat Server. Wie in Übung 1
werden serialisierte Objekte zwischen dem Clienten und Server versendet. Die Objekte 
habe ich gemäss Ihrem Vorschlag in Übung 1 abgeändert. So gibt es nur noch ein 
Antwortobjekt, welches generisch ist. Für die Exception verwende ich weiterhin eigene 
Objekte. Die Antwortobjekte sind nur soweit angepasst worden, dass sie die generischen 
Klassen verwenden können.

Ich habe ein zweites Servlet erstellt, welches ermöglicht die vorhandenen Konten anzusehen, 
wenn man das korrekte Passwort eingibt. Damit dieses Servlet (\Kursiv{WebsiteServlet}) auf 
das Bank Attribut des ersten Servlet (\Kursiv{ServerServlet}) zugreifen kann, wird in der 
\Kursiv{init} Methode des ersten Servlet die Bank als Attribute des Servlet Kontextes 
gesetzt. Dem gegnüber steht das Laden des Servlet Kontext Attributes in der \Kursiv{init} 
Methode des \Kursiv{WebsiteServlet}. Damit diese beiden Befehle in der korrekten 
Reihenfolge ausgeführt werden, ist in der Servlet Konfiguration \Kursiv{load-on-startup} 
gesetzt. Mit den numerischen Werten \Kursiv{1} bzw. \Kursiv{2} wird das nacheinander laden 
der Servlets erreicht.

Wie bereits die Übung 1, ist diese Lösung nicht Thread-Save.

\section{Server}
\lstinputlisting[language=XMLStyle,caption=Servlet Konfiguration]{../res/web.xml}
\lstinputlisting[language=java,caption=Server Servlet für Java GUI,style=JavaStyle]{../src/bank/servlet/ServerServlet.java}
\lstinputlisting[language=java,caption=Server Servlet für Browser,style=JavaStyle]{../src/bank/servlet/WebsiteServlet.java}

\section{Client}
\lstinputlisting[language=java,caption=Abstract Client Driver,style=JavaStyle]{../src/bank/communication/AbstractClientDriver.java}
\lstinputlisting[language=java,caption=Servlet Client Driver,style=JavaStyle]{../src/bank/servlet/ClientDriver.java}

\section{Kommunikation Objekte}
\subsection{Anfragen}
\lstinputlisting[language=java,caption=Interface für Anfrageobjekte,style=JavaStyle]{../src/bank/communication/request/IRequest.java}
\lstinputlisting[language=java,caption=Anfrageobjekt um Konto zu eröffnen,style=JavaStyle]{../src/bank/communication/request/CreateAccountRequest.java}
\lstinputlisting[language=java,caption=Anfrageobjekt um Konto zu schliessen,style=JavaStyle]{../src/bank/communication/request/CloseAccountRequest.java}
\lstinputlisting[language=java,caption=Anfrageobjekt um Konto abzufragen,style=JavaStyle]{../src/bank/communication/request/GetAccountRequest.java}
\lstinputlisting[language=java,caption=Anfrageobjekt um Kontonummer abzufragen,style=JavaStyle]{../src/bank/communication/request/GetAccountNumbersRequest.java}
\lstinputlisting[language=java,caption=Anfrageobjekt um Geld zu transferieren,style=JavaStyle]{../src/bank/communication/request/TransferRequest.java}
\lstinputlisting[language=java,caption=Anfrageobjekt um Kontobesitzer abzufragen,style=JavaStyle]{../src/bank/communication/request/GetOwnerRequest.java}
\lstinputlisting[language=java,caption=Anfrageobjekt um Kontostand abzufragen,style=JavaStyle]{../src/bank/communication/request/GetBalanceRequest.java}
\lstinputlisting[language=java,caption=Anfrageobjekt für Aktiv/Inaktiv Zustand,style=JavaStyle]{../src/bank/communication/request/IsActiveRequest.java}
\lstinputlisting[language=java,caption=Anfrageobjekt um Geld abzuheben,style=JavaStyle]{../src/bank/communication/request/DepositRequest.java}
\lstinputlisting[language=java,caption=Anfrageobjekt um Geld einzuzahlen,style=JavaStyle]{../src/bank/communication/request/WithdrawRequest.java}

\subsection{Antworten}
\lstinputlisting[language=java,caption=Interface für Antwortobjekte,style=JavaStyle]{../src/bank/communication/answer/IAnswer.java}
\lstinputlisting[language=java,caption=Allgemeines Antwortojekt,style=JavaStyle]{../src/bank/communication/answer/Answer.java}
\lstinputlisting[language=java,caption=Antwortobjekt für IllegalArgumentException,style=JavaStyle]{../src/bank/communication/answer/IllegalArgumentExceptionAnswer.java}
\lstinputlisting[language=java,caption=Antwortobjekt für InactiveException,style=JavaStyle]{../src/bank/communication/answer/InactiveExceptionAnswer.java}
\lstinputlisting[language=java,caption=Antwortobjekt für IOException,style=JavaStyle]{../src/bank/communication/answer/IOExceptionAnswer.java}
\lstinputlisting[language=java,caption=Antwortobjekt für OverdrawException,style=JavaStyle]{../src/bank/communication/answer/OverdrawExceptionAnswer.java}

% Inhalt Ende 
\end{document} 