\documentclass[10pt]{article}

%Math
\usepackage{amsmath}
\usepackage{amsfonts}
\usepackage{amssymb}
\usepackage{amsthm}
\usepackage{ulem}
%\usepackage{stmaryrd} %f\UTF{00FC}r Blitz!

% PageStyle
\usepackage[ngerman]{babel} % deutsche Silbentrennung
\usepackage[utf8]{inputenc}
\usepackage{fancyhdr, graphicx}
\usepackage[scaled=0.92]{helvet}
\usepackage{enumitem}
\usepackage{parskip}
\usepackage[a4paper,top=2cm]{geometry}
\setlength{\textwidth}{17cm}
\setlength{\oddsidemargin}{-0.5cm}


% Shortcommands
\newcommand{\Bold}[1]{\textbf{#1}} %Boldface
\newcommand{\Kursiv}[1]{\textit{#1}} %Italic
\newcommand{\T}[1]{\text{#1}} %Textmode
\newcommand{\Nicht}[1]{\T{\sout{$ #1 $}}} %Streicht Shit durch

%Arrows
\newcommand{\lra}{\leftrightarrow} 
\newcommand{\ra}{\rightarrow}
\newcommand{\la}{\leftarrow}
\newcommand{\lral}{\longleftrightarrow}
\newcommand{\ral}{\longrightarrow}
\newcommand{\lal}{\longleftarrow}
\newcommand{\Lra}{\Leftrightarrow}
\newcommand{\Ra}{\Rightarrow}
\newcommand{\La}{\Leftarrow}
\newcommand{\Lral}{\Longleftrightarrow}
\newcommand{\Ral}{\Longrightarrow}
\newcommand{\Lal}{\Longleftarrow}

% Code listenings
\usepackage{color}
\usepackage{xcolor}
\usepackage{listings}
\usepackage{caption}
\DeclareCaptionFont{white}{\color{white}}
\DeclareCaptionFormat{listing}{\colorbox{gray}{\parbox{\textwidth}{#1#2#3}}}
\captionsetup[lstlisting]{format=listing,labelfont=white,textfont=white}
\lstdefinestyle{JavaStyle}{
 language=Java,
 basicstyle=\footnotesize\ttfamily, % Standardschrift
 numbers=left,               % Ort der Zeilennummern
 numberstyle=\tiny,          % Stil der Zeilennummern
 stepnumber=1,              % Abstand zwischen den Zeilennummern
 numbersep=5pt,              % Abstand der Nummern zum Text
 tabsize=2,                  % Groesse von Tabs
 extendedchars=true,         %
 breaklines=true,            % Zeilen werden Umgebrochen
 frame=b,         
 %commentstyle=\itshape\color{LightLime}, Was isch das? O_o
 %keywordstyle=\bfseries\color{DarkPurple}, und das O_o
 basicstyle=\footnotesize\ttfamily,
 stringstyle=\color[RGB]{42,0,255}\ttfamily, % Farbe der String
 keywordstyle=\color[RGB]{127,0,85}\ttfamily, % Farbe der Keywords
 commentstyle=\color[RGB]{63,127,95}\ttfamily, % Farbe des Kommentars
 showspaces=false,           % Leerzeichen anzeigen ?
 showtabs=false,             % Tabs anzeigen ?
 xleftmargin=17pt,
 framexleftmargin=17pt,
 framexrightmargin=5pt,
 framexbottommargin=4pt,
 showstringspaces=false      % Leerzeichen in Strings anzeigen ?
}

%Config
\renewcommand{\headrulewidth}{0pt}
\setlength{\headheight}{15.2pt}
\let\stdsection\section
\renewcommand{\section}{\newpage\stdsection}
\usepackage{dirtree}

%Metadata
\fancyfoot[C]{}
\title{
	\vspace{5cm}
	Verteilte Systeme FS 13\\
	%\vspace{1cm}
	Übung 1
}
\author{Thomas Baumann}
\date{4. März 2013}


% hier beginnt das Dokument
\begin{document}

% Titelbild
\maketitle
\thispagestyle{fancy}

\newpage

% Inhaltsverzeichnis
\pagenumbering{Roman}
\tableofcontents	  	


\newpage
\setcounter{page}{1}
\pagenumbering{arabic}

% Inhalt Start
\section{Beschreibung}
\Bold{Ich habe die vorgegeben Klassen teilweise umbennant, damit es keine 
Probleme mit dem Verständnis der Klassennamen gibt. So beginnen alle Interface mit einem I. 
Alle Dateien sind auf der nächsten Seite aufgelistet.}

Ich habe sowohl den Client, wie auch den Server in Java programmiert. Deshalb habe ich als 
Kommunikationsart den Object-Stream gewählt.

Für jede Anfrage wird ein einzelnes Objekt erstellt und gesendet. Diese Objekte sind vom 
Interface \Kursiv{IRequest} abgeleitet, damit kann auf dem Server immer die gleiche 
Methode verwendet werden, um die Anfrage zu bearbeiten. Damit die Verarbeitung 
durchgeführt werden kann, ruft der Server die Methode \Kursiv{handleRequest(IBank)} auf 
dem erhaltenen Objekt auf und übergibt dabei seine Bank. Als Rückgabewert erhält der 
Server ein Antwortobjekt (Abgeleitet vom Interface \Kursiv{IAnswer}), welches er 
zurücksenden kann. Damit die Exceptions auch über das Netzwerk gesendet werden können, 
existieren für diese eigene Klassen.

Der Client liest das erhaltene Objekt vom \Kursiv{ObjectInputStream} und ruft die 
\Kursiv{getData()} Methode auf, diese liefert im Normalfall ein \Kursiv{Object} zurück 
oder wenn ein Fehler auftritt, wird eine Exception geworfen. Dieses Objekt wird in 
das erwartete Resultat gecastet und an das GUI weitergeleitet. Sollte nicht das korrekte 
Objekt angekommen sein, so wird eine \Kursiv{IOException} geworfen.


\section{Dateistruktur}
\dirtree{%
.1 \Kursiv{src}.
.2 \Kursiv{bank}.
.3 \Kursiv{dummy}.
.4 Driver.java.
.3 \Kursiv{gui}.
.4 \Kursiv{tests}.
.5 BankTest.java.
.5 EfficiencyTest.java.
.5 FunctionalityTest.java.
.4 BankGUI.java.
.3 \Kursiv{local}.
.4 Driver.java.
.3 \Kursiv{socket}.
.4 \Kursiv{answer}.
.5 CloseAccountAnswer.java.
.5 CreateAccountAnswer.java.
.5 GetAccountAnswer.java.
.5 GetAccountNumbersAnswer.java.
.5 GetBalanceAnswer.java.
.5 GetOwnerAnswer.java.
.5 IAnswer.java.
.5 IllegalArgumentExceptionAnswer.java.
.5 InactiveExceptionAnswer.java.
.5 IOExceptionAnswer.java.
.5 IsActiveAnswer.java.
.5 OverdrawExceptionAnswer.java.
.5 VoidAnswer.java.
.4 \Kursiv{request}.
.5 CloseAccountRequest.java.
.5 CreateAccountRequest.java.
.5 DepositRequest.java.
.5 GetAccountNumbersRequest.java.
.5 GetAccountRequest.java.
.5 GetBalanceRequest.java.
.5 GetOwnerRequest.java.
.5 IRequest.java.
.5 IsActiveRequest.java.
.5 TransferRequest.java.
.5 WithdrawRequest.java.
.4 ClientDriver.java.
.4 ServerDriver.java.
.3 Bank.java.
.3 IAccount.java.
.3 IBank.java.
.3 IBankDriver.java.
.3 InactiveException.java.
.3 IServerDriver.java.
.3 OverdrawException.java.
.3 StartClient.java.
.3 StartServer.java.
}

\section{Code: Aufgabe A}

\lstinputlisting[language=java,caption=Local Driver,style=JavaStyle]{../src/bank/local/Driver.java}
\lstinputlisting[language=java,caption=Bank Implementation,style=JavaStyle]{../src/bank/local/Bank.java}


\section{Code: Aufgabe B}
\subsection{Client}
\lstinputlisting[language=java,caption=Client Socket Driver,style=JavaStyle]{../src/bank/socket/ClientDriver.java}

\subsection{Server}
\lstinputlisting[language=java,caption=Start des Servers,style=JavaStyle]{../src/bank/StartServer.java}
\lstinputlisting[language=java,caption=Interface für Server Driver,style=JavaStyle]{../src/bank/IServerDriver.java}
\lstinputlisting[language=java,caption=Server Socket Driver,style=JavaStyle]{../src/bank/socket/ServerDriver.java}

\subsection{Kommunikation Objekte}
\subsubsection{Interface}
\lstinputlisting[language=java,caption=Interface für Request Objekte,style=JavaStyle]{../src/bank/communication/request/IRequest.java}
\lstinputlisting[language=java,caption=Interface für Answer Objekte,style=JavaStyle]{../src/bank/communication/answer/IAnswer.java}

\subsubsection{Anfrage Objekte}
\lstinputlisting[language=java,caption=Anfrage Objekt um Konto zu eröffnen,style=JavaStyle]{../src/bank/communication/request/CreateAccountRequest.java}
\lstinputlisting[language=java,caption=Anfrage Objekt um Konto zu schliessen,style=JavaStyle]{../src/bank/communication/request/CloseAccountRequest.java}
\lstinputlisting[language=java,caption=Anfrage Objekt um Konto abzufragen,style=JavaStyle]{../src/bank/communication/request/GetAccountRequest.java}
\lstinputlisting[language=java,caption=Anfrage Objekt um Kontonummer abzufragen,style=JavaStyle]{../src/bank/communication/request/GetAccountNumbersRequest.java}
\lstinputlisting[language=java,caption=Anfrage Objekt um Geld zu transferieren,style=JavaStyle]{../src/bank/communication/request/TransferRequest.java}
\lstinputlisting[language=java,caption=Anfrage Objekt um Kontobesitzer abzufragen,style=JavaStyle]{../src/bank/communication/request/GetOwnerRequest.java}
\lstinputlisting[language=java,caption=Anfrage Objekt um Kontostand abzufragen,style=JavaStyle]{../src/bank/communication/request/GetBalanceRequest.java}
\lstinputlisting[language=java,caption=Anfrage Objekt für Aktiv/Inaktiv Zustand,style=JavaStyle]{../src/bank/communication/request/IsActiveRequest.java}
\lstinputlisting[language=java,caption=Anfrage Objekt um Geld abzuheben,style=JavaStyle]{../src/bank/communication/request/DepositRequest.java}
\lstinputlisting[language=java,caption=Anfrage Objekt um Geld einzuzahlen,style=JavaStyle]{../src/bank/communication/request/WithdrawRequest.java}

\subsubsection{Antwort Objekte}
\lstinputlisting[language=java,caption=Allgemeines Antwort Objekt,style=JavaStyle]{../src/bank/communication/answer/Answer.java}
\lstinputlisting[language=java,caption=Antwort Objekt für IllegalArgumentException,style=JavaStyle]{../src/bank/communication/answer/IllegalArgumentExceptionAnswer.java}
\lstinputlisting[language=java,caption=Antwort Objekt für InactiveException,style=JavaStyle]{../src/bank/communication/answer/InactiveExceptionAnswer.java}
\lstinputlisting[language=java,caption=Antwort Objekt für IOException,style=JavaStyle]{../src/bank/communication/answer/IOExceptionAnswer.java}
\lstinputlisting[language=java,caption=Antwort Objekt für OverdrawException,style=JavaStyle]{../src/bank/communication/answer/OverdrawExceptionAnswer.java}

% Inhalt Ende 
\end{document} 