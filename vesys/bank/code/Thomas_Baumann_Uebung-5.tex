\documentclass[10pt]{article}

%Math
\usepackage{amsmath}
\usepackage{amsfonts}
\usepackage{amssymb}
\usepackage{amsthm}
\usepackage{ulem}
%\usepackage{stmaryrd} %f\UTF{00FC}r Blitz!

% PageStyle
\usepackage[ngerman]{babel} % deutsche Silbentrennung
\usepackage[utf8]{inputenc}
\usepackage{fancyhdr, graphicx}
\usepackage[scaled=0.92]{helvet}
\usepackage{enumitem}
\usepackage{parskip}
\usepackage[a4paper,top=2cm]{geometry}
\setlength{\textwidth}{17cm}
\setlength{\oddsidemargin}{-0.5cm}


% Shortcommands
\newcommand{\Bold}[1]{\textbf{#1}} %Boldface
\newcommand{\Kursiv}[1]{\textit{#1}} %Italic
\newcommand{\T}[1]{\text{#1}} %Textmode
\newcommand{\Nicht}[1]{\T{\sout{$ #1 $}}} %Streicht Shit durch

%Arrows
\newcommand{\lra}{\leftrightarrow} 
\newcommand{\ra}{\rightarrow}
\newcommand{\la}{\leftarrow}
\newcommand{\lral}{\longleftrightarrow}
\newcommand{\ral}{\longrightarrow}
\newcommand{\lal}{\longleftarrow}
\newcommand{\Lra}{\Leftrightarrow}
\newcommand{\Ra}{\Rightarrow}
\newcommand{\La}{\Leftarrow}
\newcommand{\Lral}{\Longleftrightarrow}
\newcommand{\Ral}{\Longrightarrow}
\newcommand{\Lal}{\Longleftarrow}

% Code listenings
\usepackage{color}
\usepackage{xcolor}
\usepackage{listings}
\usepackage{caption}
\DeclareCaptionFont{white}{\color{white}}
\DeclareCaptionFormat{listing}{\colorbox{gray}{\parbox{\textwidth}{#1#2#3}}}
\captionsetup[lstlisting]{format=listing,labelfont=white,textfont=white}
\lstdefinestyle{JavaStyle}{
 language=Java,
 basicstyle=\footnotesize\ttfamily, % Standardschrift
 numbers=left,               % Ort der Zeilennummern
 numberstyle=\tiny,          % Stil der Zeilennummern
 stepnumber=1,              % Abstand zwischen den Zeilennummern
 numbersep=5pt,              % Abstand der Nummern zum Text
 tabsize=2,                  % Groesse von Tabs
 extendedchars=true,         %
 breaklines=true,            % Zeilen werden Umgebrochen
 frame=b,         
 %commentstyle=\itshape\color{LightLime}, Was isch das? O_o
 %keywordstyle=\bfseries\color{DarkPurple}, und das O_o
 basicstyle=\footnotesize\ttfamily,
 stringstyle=\color[RGB]{42,0,255}\ttfamily, % Farbe der String
 keywordstyle=\color[RGB]{127,0,85}\ttfamily, % Farbe der Keywords
 commentstyle=\color[RGB]{63,127,95}\ttfamily, % Farbe des Kommentars
 showspaces=false,           % Leerzeichen anzeigen ?
 showtabs=false,             % Tabs anzeigen ?
 xleftmargin=17pt,
 framexleftmargin=17pt,
 framexrightmargin=5pt,
 framexbottommargin=4pt,
 showstringspaces=false      % Leerzeichen in Strings anzeigen ?
}
\lstdefinelanguage{XMLStyle}{
  basicstyle=\footnotesize\ttfamily,
  morestring=[b]",
  morestring=[s]{>}{<},
  morecomment=[s]{<?}{?>},
  stringstyle=\color{black},
  identifierstyle=\color[rgb]{0.0,0.0,0.6},
  keywordstyle=\color[rgb]{0.0,0.6,0.6},
  commentstyle=\color[rgb]{0.4,0.4,0.4}\upshape,
  morekeywords={xmlns,version,type}, % list your attributes here
  showstringspaces=false
}

%Config
\renewcommand{\headrulewidth}{0pt}
\setlength{\headheight}{15.2pt}
\let\stdsection\section{}
\renewcommand{\section}{\newpage\stdsection}
\usepackage{dirtree}

%Metadata
\fancyfoot[C]{}
\title{
	\vspace{5cm}
	Verteilte Systeme FS 13\\
	%\vspace{1cm}
	Übung 5
}
\author{Thomas Baumann}
\date{20. Mai 2013}


% hier beginnt das Dokument
\begin{document}

% Titelbild
\maketitle
\thispagestyle{fancy}

\newpage

% Inhaltsverzeichnis
%\pagenumbering{Roman}
%\tableofcontents	  	


\newpage
\setcounter{page}{1}
\pagenumbering{arabic}

% Inhalt Start
\section{Beschreibung}
Ich habe zwei neue Interfaces eingeführt, das einte Interface \Kursiv{IRemoteAccount} 
erweitert das Interface \Kursiv{IAccount} um das Marker Interface \Kursiv{java.rmi.Remote} 
und dementsprechend für die Bank das Interface \Kursiv{IRemoteBank}.
Die lokale Bank habe ich so abgeändert, so dass beiden Klassen die neu erstellten Interfaces
implementieren und der Rückgabetyp von \Kursiv{getAccount(String number)} ebenfalls dem neuen 
Interface entspricht. Als letzte Änderung habe ich Zeile 45 eingefügt, damit alle Anforderungen
für RMI abgedeckt sind.

Wie bereits die vorherigen Übungen, ist diese Lösung nicht Thread-Safe.

\section{Code}
\lstinputlisting[language=java,caption=Interface des Accounts für RMI,style=JavaStyle]{../src/bank/IRemoteAccount.java}
\lstinputlisting[language=java,caption=Interface der Bank für RMI,style=JavaStyle]{../src/bank/IRemoteBank.java}

\lstinputlisting[language=java,caption=Lokale Bank,style=JavaStyle]{../src/bank/local/Bank.java}

\lstinputlisting[language=java,caption=Server Driver,style=JavaStyle]{../src/bank/rmi/ServerDriver.java}
\lstinputlisting[language=java,caption=Client Driver,style=JavaStyle]{../src/bank/rmi/ClientDriver.java}

% Inhalt Ende 
\end{document} 
