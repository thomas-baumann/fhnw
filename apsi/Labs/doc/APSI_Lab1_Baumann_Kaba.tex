\documentclass[12pt]{scrartcl}
\usepackage[utf8]{inputenc}
 \usepackage{fancyhdr, graphicx}
 \usepackage[german]{babel}
 \usepackage[scaled=0.92]{helvet}
 \usepackage{enumitem}
 \usepackage{parskip}
 \usepackage{lastpage} % for getting last page number
\usepackage{listings}
\usepackage{caption}
\usepackage{color}
\usepackage{xcolor}
\DeclareCaptionFont{white}{\color{white}}
\DeclareCaptionFormat{listing}{\colorbox{gray}{\parbox{\textwidth}{#1#2#3}}}
\newcommand\Fontvi{\fontsize{6}{7.2}\selectfont}

 \renewcommand{\familydefault}{\sfdefault}
 
 \fancypagestyle{firststyle}{ %Style of the first page
 \fancyhf{}
 \fancyheadoffset[L]{0.6cm}
 \lhead{
 \includegraphics[scale=0.8]{./FHNW_HT_10mm.jpg}}
 \renewcommand{\headrulewidth}{0pt}
 \cfoot{Thomas Baumann \& Egemen Kaba}
}

\fancypagestyle{documentstyle}{ %Style of the rest of the document
 \fancyhf{}
 \fancyheadoffset[L]{0.6cm}
\lhead{
 \includegraphics[scale=0.8]{./FHNW_HT_10mm.jpg}}
 \renewcommand{\headrulewidth}{0pt}
 \lfoot{Laborübung 1}
 \cfoot{Thomas Baumann \& Egemen Kaba}
 \rfoot{\thepage\ / \pageref{LastPage} }
}

\pagestyle{firststyle} %different look of first page
 
\title{ %Titel
Applikationssicherheit 
\vspace{0.2cm}
}

 \begin{document}
 \maketitle
 \thispagestyle{firststyle}
 \pagestyle{firststyle}
 \begin{abstract}
 \begin{center}
 Laborübung 1
 \end{center}
 \vspace{0.5cm}
\hrulefill
\end{abstract}

 \pagestyle{documentstyle}
 \tableofcontents
 \pagebreak
\section{Ressourcen}
Die Dateien wurden in ISO-8859-1 kodiert, damit die Umlaute korrekt dargestellt und ausgelesen werden können.

\textbf{original.txt}\newline
Die originale Textnachricht von Bob als Plaintext.

\textbf{Patterns.txt}\newline
Patterns, die benutzt werden, um eine gefälschte Nachricht zu generieren, welche den gleichen Hashwert, wie die Originalnachricht hat.\newline
Die Form dazu ist folgende:
\begin{itemize}
\item Je Pattern existiert eine Linie in der Datei
\begin{itemize}
\item Gesamthaft also 32 Linien $\Longrightarrow$ Bits in einem Integer
\end{itemize}
\item Je Linie existieren zwei Möglichkeiten, die durch eine vertikale Trennlinie unterteilt sind (Text 1 $|$Text 2 ) $\Longrightarrow$ 0 bzw. 1 eines Bits
\end{itemize}

\textbf{Patterns\_Org.txt}\newline
Patterns, um eine Originalnachricht zu erstellen.\newline
Sie ist gleich aufgebaut, wie \textit{Patterns.txt}.

\section{Klassen}
Implementierungsdetails siehe Sourcecode.

\textbf{Start}\newline
Dies ist ein möglicher Startpunkt der Applikation. Dabei wird die vorgegebene originale Nachricht verwendet und permutierende Fälschungen erstellt.

\textbf{StartPermutation (main)}\newline
Dies ist der zweite  mögliche Startpunkt der Applikation. In diesem Fall werden sowohl permutierende Fälschungen, als auch permutierende originale Nachrichten erstellt. Dabei werden zwei verschiedene Texte pro Aufruf erstellt.

\textbf{TextGenerator}\newline
Diese Klasse liest ein Pattern-Datei ein und generiert daraus bei Aufruf der entsprechenden Methode Texte. Da ein Pattern-Datei genau 32 Möglichkeiten hat ein Text zu generieren, wird auf Basis eines Integers gearbeitet, um zu bestimmen, welche Möglichkeit (0 oder 1 => links oder rechts) auszuwählen ist. Dieser Integer kann entweder inkrementell oder randomisiert gewählt werden.

\textbf{DESHash}\newline
In dieser Klasse ist die DES-Funktion aus der Aufgabenstellung implementiert. Dabei werden die Klassen DESEngine für den DES und PaddedBufferedBlockCipher, um die Engine in einer BlockCipher auszuführen, von Bouncy Castle verwendet.

Es wurde die Methodik der Aufgabenstellung übernommen. Einzige Anmerkung gilt es bei dem Ergebnis des DES zu machen. Und zwar arbeitet DES mit zwei 64 Bit Blöcken. Bei der Weiterverwendung des Outputs als ein 64 Bit Block müssen diese zwei Blöcke dementsprechend mit xor vereint werden.

\section{Ergebnisse}
Es wurden zwei Varianten bzw. Interpretationsmöglichkeiten der Aufgabestellung implementiert.

\textbf{Variante 1}\newline
Die erste Variante sieht vor, dass ein festes originales File vorhanden ist. Die Fälschungen werden anhand eines Pattern-Files generiert.\newline
Als Startmethode wird hier die Klasse Start genommen.
Diese Variante erzielte jedoch kein Ergebnis, auch wenn sie über Nacht, also mehrere Stunden, lief.

\textbf{Variante 2}\newline
Die zweite Variante sieht vor, dass Originale, sowie Fälschungen anhand je eines Pattern-Files generiert werden.\newline
Als Startmethode wird hier die Klasse StartPermutation genommen und die gewünschte Anzahl Kollisionen, nach der gesuchten werden soll, kann in der Variable count angegeben werden.\newline
Die Laufzeit beträgt ca. 5 Minuten für fünf Kollisionen. Dies kann jedoch variieren, da mit der randomisierten Textgenerierung früher oder später ein entsprechender Text erstellt wird.

Diese Variante erzielte folgendes Ergebnis mit 5 Kollisionen:

--------------------------------------------\newline
Found Hash: 574314342\newline
Strings mapping to that Hash: \newline
1) Meine liebe Alice, ich bedanke mich vom Herzen für Deinen sehr willkommenen Auftrag. Ich möchte Dich vertraulich aufmerksam machen, dass unsere Ingenieure etwas Sensationelles in den Kächern  halten : SunShineForever. Du kannst Dir plastisch denken was dahinten steckt! Du wirst die einzige sein, eine ansehnliche Menge Muster kostenlos zu erhalten. Ich bitte Dich den Betrag von 100.000.- Schweizer Franken auf das Konto mit der Nr. 222-1101.461.12 der  Bank ABC AG, in Basel CH zu überweisen.Ich freue mich Dir geholfen zu haben und verbleibe ich freundlichen Grüssen. Dein Bob, CEO\newline
2) liebe Alice, ich bedanke mich aufrichtig für den erfreulichen Lieferungsvertrag. Ich möchte Dich aufmerksam machen, dass unsere Ingenieure etwas Sensationelles in den Reagenzgläsern  halten : SunShineForever. Du kannst Dir plastisch vorstellen was dahinten steckt! Du wirst die erste  sein, eine ansehnliche Anzahl Fläschen gratis zu erhalten. Ich bitte Dich die Summe von 100.000.- CHF auf das Konto mit der Nr. 222-1101.461.10 von der Bank ABC AG, in Basel zu überweisen.Ich freue mich Dir gedient zu haben und schliesse ich freundlichen Grüssen. Bob, CEO\newline
--------------------------------------------\newline
Found Hash: -2128439365\newline
Strings mapping to that Hash: \newline
1) liebe Alice, ich bedanke mich vom Herzen für den sehr willkommenen Auftrag. Ich möchte Deine Firma vertraulich aufmerksam machen, dass wir etwas Herausragendes in den Reagenzgläsern  halten : SunShineForever. Du kannst Dir  vorstellen was dahinten steckt! Du wirst die erste  sein, eine ansehnliche Anzahl Muster gratis zu kriegen. Ich bitte Dich die Summe von 100.000.- Schweizer Franken auf das Konto mit der Nr. 222-1101.461.12 der  Bank ABC AG, 4001 Basel zu überweisen.Ich freue mich Dir geholfen zu haben und verbleibe ich mit lieben Grüssen. Dein Bob, Geschäftsführer\newline
2) Meine liebe Alice, ich bedanke mich aufrichtig für den erfreulichen Lieferungsvertrag. Ich möchte Dich vertraulich aufmerksam machen, dass unsere Ingenieure etwas Herausragendes in den Kächern haben: SunShineForever. Du kannst Dir plastisch vorstellen was darin steckt! Du wirst die erste  sein, eine grosse Menge Muster kostenlos zu kriegen. Ich bitte Dich den Betrag von 100.000.- Schweizer Franken auf das Konto Nr. 222-1101.461.10 von der Bank ABC AG, in Basel CH zu überweisen.Ich hoffe Dir geholfen zu haben und verbleibe ich freundlichen Grüssen. Dein Bob, CEO\newline
--------------------------------------------\newline
Found Hash: 1990800991\newline
Strings mapping to that Hash: \newline
1) Meine liebe Alice, ich bedanke mich vom Herzen für Deinen sehr willkommenen Auftrag. Ich möchte Deine Firma aufmerksam machen, dass unsere Ingenieure etwas Sensationelles in den Kächern  halten : SunShineForever. Du kannst Dir  vorstellen was darin steckt! Du wirst die erste  sein, eine grosse Menge Fläschen gratis zu kriegen. Ich bitte Dich den Betrag von 100.000.- CHF auf das Konto mit der Nr. 222-1101.461.12 der  Bank ABC AG, in Basel CH zu überweisen.Ich hoffe Dir geholfen zu haben und schliesse ich mit lieben Grüssen. Dein Bob, Geschäftsführer\newline
2) liebe Alice, ich bedanke mich aufrichtig für Deinen erfreulichen Auftrag. Ich möchte Dich aufmerksam machen, dass unsere Ingenieure etwas Herausragendes in den Kächern  halten : SunShineForever. Du kannst Dir plastisch vorstellen was darin steckt! Du wirst die einzige sein, eine grosse Menge Fläschen gratis zu erhalten. Ich bitte Dich die Summe von 100.000.- Schweizer Franken auf das Konto Nr. 222-1101.461.10 von der Bank ABC AG, 4001 Basel CH zu überweisen.Ich freue mich Dir geholfen zu haben und verbleibe ich freundlichen Grüssen. Bob, CEO\newline
--------------------------------------------\newline
Found Hash: 252380\newline
Strings mapping to that Hash: \newline
1) Meine liebe Alice, ich bedanke mich aufrichtig für den erfreulichen Auftrag. Ich möchte Dich aufmerksam machen, dass wir etwas Herausragendes in den Kächern haben: SunShineForever. Du kannst Dir  vorstellen was dahinten steckt! Du wirst die erste  sein, eine grosse Menge Fläschen kostenlos zu kriegen. Ich bitte Dich den Betrag von 100.000.- CHF auf das Konto Nr. 222-1101.461.12 von der Bank ABC AG, 4001 Basel CH zu überweisen.Ich hoffe Dir gedient zu haben und verbleibe ich freundlichen Grüssen. Bob, CEO\newline
2) liebe Alice, ich bedanke mich aufrichtig für Deinen erfreulichen Auftrag. Ich möchte Dich vertraulich aufmerksam machen, dass wir etwas Herausragendes in den Kächern  halten : SunShineForever. Du kannst Dir  vorstellen was dahinten steckt! Du wirst die erste  sein, eine ansehnliche Menge Muster kostenlos zu kriegen. Ich bitte Dich die Summe von 100.000.- CHF auf das Konto Nr. 222-1101.461.10 von der Bank ABC AG, in Basel CH zu überweisen.Ich freue mich Dir geholfen zu haben und schliesse ich freundlichen Grüssen. Bob, CEO\newline
--------------------------------------------\newline
Found Hash: 1391400107\newline
Strings mapping to that Hash: \newline
1) Meine liebe Alice, ich bedanke mich aufrichtig für Deinen erfreulichen Lieferungsvertrag. Ich möchte Deine Firma vertraulich aufmerksam machen, dass wir etwas Sensationelles in den Reagenzgläsern haben: SunShineForever. Du kannst Dir plastisch denken was dahinten steckt! Du wirst die einzige sein, eine ansehnliche Menge Muster gratis zu kriegen. Ich bitte Dich die Summe von 100.000.- Schweizer Franken auf das Konto Nr. 222-1101.461.12 von der Bank ABC AG, in Basel CH zu überweisen.Ich hoffe Dir geholfen zu haben und schliesse ich freundlichen Grüssen. Dein Bob, Geschäftsführer\newline
2) liebe Alice, ich bedanke mich aufrichtig für den sehr willkommenen Lieferungsvertrag. Ich möchte Dich vertraulich aufmerksam machen, dass wir etwas Sensationelles in den Reagenzgläsern haben: SunShineForever. Du kannst Dir plastisch vorstellen was darin steckt! Du wirst die einzige sein, eine ansehnliche Anzahl Muster gratis zu kriegen. Ich bitte Dich die Summe von 100.000.- Schweizer Franken auf das Konto Nr. 222-1101.461.10 der  Bank ABC AG, 4001 Basel zu überweisen.Ich freue mich Dir geholfen zu haben und verbleibe ich mit lieben Grüssen. Dein Bob, Geschäftsführer\newline
--------------------------------------------


 \end{document}